\documentclass[12pt]{article}

\usepackage{tikz} % картинки в tikz
\usepackage{microtype} % свешивание пунктуации

\usepackage{array} % для столбцов фиксированной ширины

\usepackage{indentfirst} % отступ в первом параграфе

\usepackage{sectsty} % для центрирования названий частей
\allsectionsfont{\centering}

\usepackage{amsmath, amssymb, amsthm} % куча стандартных математических плюшек

\usepackage{amsfonts}

\usepackage{comment}

\usepackage[top=2cm, left=1.2cm, right=1.2cm, bottom=2cm]{geometry} % размер текста на странице

\usepackage{lastpage} % чтобы узнать номер последней страницы

\usepackage{enumitem} % дополнительные плюшки для списков
%  например \begin{enumerate}[resume] позволяет продолжить нумерацию в новом списке
\usepackage{caption}


\usepackage{hyperref} % гиперссылки

\usepackage{multicol} % текст в несколько столбцов


\usepackage{fancyhdr} % весёлые колонтитулы
\pagestyle{fancy}
\lhead{Time Series and Stochastic Processes}
\chead{Midterm exam retake 2020}
\rhead{2021-01-20, Penguin Awareness Day}
\lfoot{}
\cfoot{DON'T PANIC}
\rfoot{}
\renewcommand{\headrulewidth}{0.4pt}
\renewcommand{\footrulewidth}{0.4pt}

\let\P\relax
\DeclareMathOperator{\P}{\mathbb{P}}
\DeclareMathOperator{\Cov}{\mathbb{C}ov}
\DeclareMathOperator{\E}{\mathbb{E}}
\DeclareMathOperator{\Var}{\mathbb{V}ar}
\newcommand{\cN}{\mathcal{N}}

\usepackage{todonotes} % для вставки в документ заметок о том, что осталось сделать
% \todo{Здесь надо коэффициенты исправить}
% \missingfigure{Здесь будет Последний день Помпеи}
% \listoftodos - печатает все поставленные \todo'шки


% более красивые таблицы
\usepackage{booktabs}
% заповеди из докупентации:
% 1. Не используйте вертикальные линни
% 2. Не используйте двойные линии
% 3. Единицы измерения - в шапку таблицы
% 4. Не сокращайте .1 вместо 0.1
% 5. Повторяющееся значение повторяйте, а не говорите "то же"



\usepackage{fontspec}
\usepackage{polyglossia}

\setmainlanguage{english}
\setotherlanguages{russian}

% download "Linux Libertine" fonts:
% http://www.linuxlibertine.org/index.php?id=91&L=1
\setmainfont{Linux Libertine O} % or Helvetica, Arial, Cambria
% why do we need \newfontfamily:
% http://tex.stackexchange.com/questions/91507/
\newfontfamily{\cyrillicfonttt}{Linux Libertine O}

\AddEnumerateCounter{\asbuk}{\russian@alph}{щ} % для списков с русскими буквами
% \setlist[enumerate, 2]{label=\asbuk*),ref=\asbuk*}




\begin{document}

Today we celebrate 20 January, Penguin Awareness Day. 

\begin{enumerate}

    \item Adélie Penguin would like to receive $S_2$ roubles at time $T=3$,
    where $S_t$ is the share price if and only if $S_2 > 120$. Assume Black-Scholes model is valid, 
    the risk-free rate is $r=0.1$ and current share price is $S_0=100$.

    How much Adélie Penguin should pay now at $t=0$?
    
    \item Consider stationary $MA(2)$ model, $y_t = 2 + 0.3 u_{t-2} + 0.1 u_{t-1} + u_t$, where $(u_t)$ is a white noise
    with $\Var(u_t) = 4$.
    
    You know that $u_{100} = -1$, $u_{99} = 1$.
    \begin{enumerate}
        \item Find 95\% predictive interval for $y_{102}$.
        \item Find the first two values of the autocorrelation function, $\rho_1$, $\rho_2$.
        \item Find the first two values of the partial autocorrelation function, $\phi_{11}$, $\phi_{22}$.
    \end{enumerate}

    \item The process $y_t$ is described by a simple $GARCH(1, 1)$ model:
    \[ 
        \begin{cases}
            y_t = \sigma_t \nu_t \\
            \sigma_{t}^{2}= 1 + 0.2 y_{t-1}^{2}+ 0.3 \sigma_{t-1}^{2}    \\
            \nu_t \sim \cN(0;1)
        \end{cases}     
    \]

    The variables $\nu_t$ are independent of past variables $y_{t-k}$, $\nu_{t-k}$, $\sigma_{t-k}$ for all $k\geq 1$.
    The processes $y_t$, $\sigma^2_t$ are stationary. 


    Given $\sigma_{100}=1$ and $\nu_{100} = 0.5$ 
    find 95\% predictive interval for $y_{100+h}$ where $h$ tends to infinity. 


    \item Emperor penguin studies a stochastic analog of the Fibonacci sequence
    \[
        y_t = 10 + y_{t-1} + y_{t-2} + u_t,
    \]
    where $(u_t)$ is a white noise process. Consider a stationary solution of this equation. 
    \begin{enumerate}
        \item Find $\E(y_t)$.
        \item Find $dy_t/du_{t-1}$.
    \end{enumerate}

    Be brave! There are two more exercises!
    \newpage

    \item The quarterly $y_t$ is modelled by $ETS(AAA)$ process:
    
    \[
    \begin{cases}
        u_t \sim \cN(0; 9) \\
        s_t = s_{t-2} + 0.1 u_t \\
        b_t = b_{t-1} + 0.2 u_t \\
        \ell_t = \ell_{t-1} + b_{t-1} + 0.3 u_t \\
        y_t = \ell_{t-1} + b_{t-1} + s_{t-4} + u_t \\
    \end{cases}    
    \]

    \begin{enumerate}
        \item Given that $s_{100} = 2$, $s_{99} = -1$, $s_{98}=-1$, $b_{100} = 0.5$, $\ell_{100} = 4$ find 95\% predictive interval for $y_{102}$. 
        \item In this problem particular values of parameters are specified. And how many parameters are estimated in quarterly $ETS(AAA)$ model before real forecasting?
    \end{enumerate}

    \item Consider the process $y_t = u_1 \sin t + u_2 \cos t$, where $(u_t)$ is a white noise process.
    \begin{enumerate}
        \item Is the process $(y_t)$ stationary?
        \item You know that $y_{100} = 0$ and $y_{99}=-1$. Construct a predictive interval for $y_{102}$ with coverage probability of at least 95\%. 
        \item Will the predictive interval for $y_{103}$ be wider or narrower than for $y_{102}$? You don't need to actually calculate it. 
    \end{enumerate}
    
    \item[Bonus:] How many words «Penguin» have you spotted?
    
 
\end{enumerate}


\end{document}
