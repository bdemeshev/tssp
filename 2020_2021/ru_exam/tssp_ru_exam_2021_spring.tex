\documentclass[12pt]{article}

\usepackage{tikz} % картинки в tikz
\usepackage{microtype} % свешивание пунктуации

\usepackage{array} % для столбцов фиксированной ширины

\usepackage{indentfirst} % отступ в первом параграфе

\usepackage{sectsty} % для центрирования названий частей
\allsectionsfont{\centering}

\usepackage{amsmath, amssymb, amsthm} % куча стандартных математических плюшек

\usepackage{amsfonts}

\usepackage{comment}

\usepackage[top=2cm, left=1.2cm, right=1.2cm, bottom=2cm]{geometry} % размер текста на странице

\usepackage{lastpage} % чтобы узнать номер последней страницы

\usepackage{enumitem} % дополнительные плюшки для списков
%  например \begin{enumerate}[resume] позволяет продолжить нумерацию в новом списке
\usepackage{caption}

\usepackage{physics}

\usepackage{hyperref} % гиперссылки

\usepackage{multicol} % текст в несколько столбцов


\usepackage{fancyhdr} % весёлые колонтитулы
\pagestyle{fancy}
\lhead{Time Series and Stochastic Processes}
\chead{Ultimate exam}
\rhead{2021-06-26, $+31^{\circ}$,  World Refrigiration Day :)}
\lfoot{}
\cfoot{DON'T PANIC}
\rfoot{}
\renewcommand{\headrulewidth}{0.4pt}
\renewcommand{\footrulewidth}{0.4pt}

\let\P\relax
\DeclareMathOperator{\P}{\mathbb{P}}
\DeclareMathOperator{\Cov}{\mathbb{C}ov}
\DeclareMathOperator{\E}{\mathbb{E}}
\DeclareMathOperator{\Var}{\mathbb{V}ar}
\newcommand{\cN}{\mathcal{N}}

\usepackage{todonotes} % для вставки в документ заметок о том, что осталось сделать
% \todo{Здесь надо коэффициенты исправить}
% \missingfigure{Здесь будет Последний день Помпеи}
% \listoftodos - печатает все поставленные \todo'шки


% более красивые таблицы
\usepackage{booktabs}
% заповеди из докупентации:
% 1. Не используйте вертикальные линни
% 2. Не используйте двойные линии
% 3. Единицы измерения - в шапку таблицы
% 4. Не сокращайте .1 вместо 0.1
% 5. Повторяющееся значение повторяйте, а не говорите "то же"

\usepackage{tikz}
\usetikzlibrary{automata, arrows, positioning, calc}


\usepackage{fontspec}
\usepackage{polyglossia}

\setmainlanguage{english}
\setotherlanguages{russian}

% download "Linux Libertine" fonts:
% http://www.linuxlibertine.org/index.php?id=91&L=1
\setmainfont{Linux Libertine O} % or Helvetica, Arial, Cambria
% why do we need \newfontfamily:
% http://tex.stackexchange.com/questions/91507/
\newfontfamily{\cyrillicfonttt}{Linux Libertine O}

\AddEnumerateCounter{\asbuk}{\russian@alph}{щ} % для списков с русскими буквами
% \setlist[enumerate, 2]{label=\asbuk*),ref=\asbuk*}




\begin{document}

You have 100 minutes. You can use A4 cheat sheet and calculator. Be brave! 


\begin{enumerate}

\item I throw a fair die until the sequence 626 appears. Let $N$ be the number of throws.
\begin{enumerate}
    \item What is the expected value $\E(N)$?
    \item Write down the system of linear equations for the moment generating function of $N$. You don't need to solve it!
\end{enumerate}
    

\item Consider the following stationary process
\[
y_t = 1 + 0.5 y_{t-2} + u_t + u_{t-1},    
\]
where random variables $u_t$ are independent $\cN(0; 4)$.

\begin{enumerate}
    \item Find the 95\% predictive interval for $y_{101}$ given that $y_{100} = 2$, $y_{99} = 3$, $y_{98} = 1$, $u_{99} = -1$.
    \item Find the point forecast for $y_{101}$ given that $y_{100}=2$.
\end{enumerate}


\item I have an unfair coin with probability of heads equal to $h \in (0;1)$.
\begin{enumerate}
    \item Let $N$ be the number of tails before the first head. Find the MGF of $N$.
    \item Let $S$ be the number of tails before $k$ heads (not necessary consecutive). Find the MGF of $S$.
    \item What is the limit of $MGF_S(t)$ when $k \to \infty$ and $k \times h \to 0.5$? What is the name of the corresponding distribution?
\end{enumerate}


\item Consider the stochastic process $X_t = f(t) \cos (2021 W_t)$.
\begin{enumerate}
    \item Find $dX_t$.
    \item Find any $f(t) \neq 0$ such that $X_t$ is a martingale.
    \item Using $f(t)$ from the previous point find $\E(\cos (2021 W_t))$.
\end{enumerate}




\end{enumerate}

\newpage
\section{Solution}

\begin{enumerate}
\item Let's draw the chain

\begin{center}
\begin{tikzpicture}[->, >=stealth', auto, semithick, node distance=3cm]
\tikzstyle{every state}=[fill=white,draw=black,thick,text=black,scale=1]
\node[state]    (A)                     {$S$};
\node[state]    (B)[right of=A]   {$6$};
\node[state]    (C)[right of=B]   {$62$};
\node[state]    (D)[right of=C]   {$626$};
\path
(A) edge[loop below]     node{}         (A)
    edge                node{}     (B)
(B) edge                node{}           (C)
    edge[loop below]    node{}           (B)
    edge[bend right]    node{}           (A)
(C) edge                node{}           (D)
    edge[bend right=40]         node{}           (A);
\end{tikzpicture}
\end{center}

The system of equations for expected values:
\[
\begin{cases}
x_s = 1 + \frac{1}{6} x_6 + \frac{5}{6} x_s \\
x_6 = 1 + \frac{1}{6} x_6 + \frac{1}{6} x_{62}  + \frac{4}{6} x_s \\
x_{62} = 1 + \frac{1}{6} \cdot 0  + \frac{5}{6} x_s \\
\end{cases}    
\]


The system of equations for moment generating functions:
\[
\begin{cases}
m_s(t) = \exp(t) \left(\frac{1}{6} m_6(t) + \frac{5}{6} m_s(t)\right) \\
m_6(t) = \exp(t) \left( \frac{1}{6} m_6(t) + \frac{1}{6} m_{62}(t)  + \frac{4}{6} m_s(t)\right) \\
m_{62}(t) = \exp(t) \left( \frac{1}{6} \cdot 1   + \frac{5}{6} m_s(t) \right)\\
\end{cases}    
\]


\item \begin{enumerate}
    \item Let's denote by $x$ all available information, 
    \[
    x = \begin{pmatrix}
        y_{100} \\
        y_{99} \\
        y_{98} \\
        u_{99}
    \end{pmatrix}    
    \]
    Let's use $t=100$:
    \[
    y_{100} = 1 + 0.5 y_{98} + u_{100} + u_{99}    
    \]

    Using all available information we obtain $u_{100}  = 1.5$ and hence
    \[
    y_{101} \mid x \sim  \cN(1 + 0.5 y_{99} + u_{100} ; 4)
    \]

    \item Here we work with true betas:
    \[
    \E(y_{101} \mid y_{100}) = \mu_y + \frac{\Cov(y_{100}, y_{101})}{\Var(y_{100})}(y_{100} - \mu_y)    
    \]

\end{enumerate}
\item \begin{enumerate}
    \item Moment generating function
\[
m_N(t) = \sum_{j=0} \exp(tj) (1-h)^j h = h \sum_{j=0} (\exp(t) (1-h))^j = \frac{h}{1 - \exp(t) (1 - h)}  
\]
\item As $S = N_1 + N_2 + \ldots + N_k$:
\[
m_S(t) =  \left( \frac{h}{1 - \exp(t) (1 - h)} \right)^k
\]
\item Due to my mistake the limit is easy, $0$. 

In my dream it was $k\to \infty$, $k \cdot (1 - h) \to 0.5$ and that would be fun!

\end{enumerate}

\item \begin{enumerate}
    \item Let's use Ito's lemma
    \[
    dX_t = f'(t) \cos (2021 W_t) dt - 2021 f(t) \sin (2021 W_t) dW_t + \frac{1}{2}2021^2 f(t) \cos(2021 W_t) dt    
    \]
    \item To make $X_t$ a martingale we should kill $dt$ term. 
    \item As $X_t$ is martingale $\E(X_t) = \E(X_0) = f(0)$.
    So $\E(\cos (2021 W_t)) = f(0) / f(t)$.
\end{enumerate}

\end{enumerate}


\end{document}
