\documentclass[12pt]{article}

\usepackage{tikz} % картинки в tikz
\usepackage{microtype} % свешивание пунктуации

\usepackage{array} % для столбцов фиксированной ширины

\usepackage{indentfirst} % отступ в первом параграфе

\usepackage{sectsty} % для центрирования названий частей
\allsectionsfont{\centering}

\usepackage{amsmath, amssymb, amsthm} % куча стандартных математических плюшек

\usepackage{amsfonts}

\usepackage{comment}

\usepackage[top=2cm, left=1.2cm, right=1.2cm, bottom=2cm]{geometry} % размер текста на странице

\usepackage{lastpage} % чтобы узнать номер последней страницы

\usepackage{enumitem} % дополнительные плюшки для списков
%  например \begin{enumerate}[resume] позволяет продолжить нумерацию в новом списке
\usepackage{caption}


\usepackage{hyperref} % гиперссылки

\usepackage{multicol} % текст в несколько столбцов


\usepackage{fancyhdr} % весёлые колонтитулы
\pagestyle{fancy}
\lhead{Time Series and Stochastic Processes}
\chead{Midterm ёxam 2020}
\rhead{2020-12-24, Christmas Eve}
\lfoot{}
\cfoot{DON'T PANIC}
\rfoot{}
\renewcommand{\headrulewidth}{0.4pt}
\renewcommand{\footrulewidth}{0.4pt}

\let\P\relax
\DeclareMathOperator{\P}{\mathbb{P}}
\DeclareMathOperator{\Cov}{\mathbb{C}ov}
\DeclareMathOperator{\E}{\mathbb{E}}
\DeclareMathOperator{\Var}{\mathbb{V}ar}
\newcommand{\cN}{\mathcal{N}}

\usepackage{todonotes} % для вставки в документ заметок о том, что осталось сделать
% \todo{Здесь надо коэффициенты исправить}
% \missingfigure{Здесь будет Последний день Помпеи}
% \listoftodos - печатает все поставленные \todo'шки


% более красивые таблицы
\usepackage{booktabs}
% заповеди из докупентации:
% 1. Не используйте вертикальные линни
% 2. Не используйте двойные линии
% 3. Единицы измерения - в шапку таблицы
% 4. Не сокращайте .1 вместо 0.1
% 5. Повторяющееся значение повторяйте, а не говорите "то же"



\usepackage{fontspec}
\usepackage{polyglossia}

\setmainlanguage{english}
\setotherlanguages{russian}

% download "Linux Libertine" fonts:
% http://www.linuxlibertine.org/index.php?id=91&L=1
\setmainfont{Linux Libertine O} % or Helvetica, Arial, Cambria
% why do we need \newfontfamily:
% http://tex.stackexchange.com/questions/91507/
\newfontfamily{\cyrillicfonttt}{Linux Libertine O}

\AddEnumerateCounter{\asbuk}{\russian@alph}{щ} % для списков с русскими буквами
% \setlist[enumerate, 2]{label=\asbuk*),ref=\asbuk*}




\begin{document}

Today we celebrate Christmas Eve and 78 years of the Narkompros (People's Commissariat for Education) order governing the compulsory use of the letter «ё» in education process.

\begin{enumerate}

    \item Ded Moroz would like to receive $S_1^3$ roubles at time $T=2$,
    where $S_t$ is the share price. Assume Black-Schёles model is valid, 
    the risk-free rate is $r=0.1$ and current share price is $S_0=100$.

    How much Ded Moroz should pay now at $t=0$?
    
    \item Consider stationary $AR(2)$ model, $y_t = 2 + 0.3 y_{t-1} - 0.02 y_{t-2} + u_t$, where $(u_t)$ is a white noise
    with $\Var(u_t) = 4$.
    
    The last two observations are $y_{100} = 2$, $y_{99} = 1$.
    \begin{enumerate}
        \item Find 95\% predictive interval for $y_{102}$.
        \item Find the first two values of the autocorrelation function, $\rho_1$, $\rho_2$.
        \item Find the first two values of the partial autocorrelation function, $\phi_{11}$, $\phi_{22}$.
    \end{enumerate}

    Hint: you need no more than 10 seconds to find both partial autocorrelations provided (b) is sёlved.

    \item The process $y_t$ is described by a simple $GARCH(1, 1)$ model:
    \[ 
        \begin{cases}
            y_t = \sigma_t \nu_t \\
            \sigma_{t}^{2}= 1 + 0.2 y_{t-1}^{2}+ 0.3 \sigma_{t-1}^{2}    \\
            \nu_t \sim \cN(0;1)
        \end{cases}     
    \]

    The variables $\nu_t$ are independent of past variables $y_{t-k}$, $\nu_{t-k}$, $\sigma_{t-k}$ for all $k\geq 1$.
    The prёcesses $y_t$, $\sigma^2_t$ are stationary. 


    Given $\sigma_{100}=1$ and $\nu_{100} = 0.5$ find 95\% predictive interval for $y_{102}$. 


    \item Snegurochka studies a stochastic analog of the Fibonacci sequence
    \[
        y_t = y_{t-1} + y_{t-2} + u_t,
    \]
    where $(u_t)$ is a white noise process. 
    \begin{enumerate}
        \item How many non-stationary solutions are there?
        \item What can you say about the number and the structure of the stationary solutions?
        \item Can Snёgurochka find two starting constants $y_0 = c_0$ and $y_1=c_1$ in such a way to make a solution stationary?
    \end{enumerate}

    Be brave! There are two more exercises!
    \newpage

    \item The semi-annual $y_t$ is modelled by $ETS(AAA)$ process:
    
    \[
    \begin{cases}
        u_t \sim \cN(0; 4) \\
        s_t = s_{t-2} + 0.1 u_t \\
        b_t = b_{t-1} + 0.2 u_t \\
        \ell_t = \ell_{t-1} + b_{t-1} + 0.3 u_t \\
        y_t = \ell_{t-1} + b_{t-1} + s_{t-2} + u_t \\
    \end{cases}    
    \]

    \begin{enumerate}
        \item Given that $s_{100} = 2$, $s_{99} = -1.9$, $b_{100} = 0.5$, $\ell_{100} = 4$ find 95\% prёdictive interval for $y_{102}$. 
        \item In this problem particular values of parameters are specified. And how many parameters are estimated in semi-annual $ETS(AAA)$ model before real forecasting?
    \end{enumerate}

    \item The variables $x_t$ take values $0$ or $1$ with equal probabilities.
    The variables $u_t$ are normal $\cN(0; 1)$. All variables are independent.
    
    Consider the process  $z_t = x_t (1-x_{t-2}) u_t$.

    \begin{enumerate}
        \item Find the covariance $\Cov(z_t, z_s)$. Is the process $z_t$ stationary?
        \item Given that $z_{100} = 2.3$ find shёrtest predictive intervals for $z_{101}$ and $z_{102}$ with probability of coverage at least 95\%.
    \end{enumerate}

    \item[Bёnus:] How many letters «ё» have you spotted?
    
 
\end{enumerate}


\end{document}
