\documentclass[12pt]{article}

\usepackage{tikz} % картинки в tikz
\usepackage{microtype} % свешивание пунктуации

\usepackage{array} % для столбцов фиксированной ширины

\usepackage{indentfirst} % отступ в первом параграфе

\usepackage{sectsty} % для центрирования названий частей
\allsectionsfont{\centering}

\usepackage{amsmath, amssymb, amsthm} % куча стандартных математических плюшек

\usepackage{amsfonts}

\usepackage{comment}

\usepackage[top=2cm, left=1.2cm, right=1.2cm, bottom=2cm]{geometry} % размер текста на странице

\usepackage{lastpage} % чтобы узнать номер последней страницы

\usepackage{enumitem} % дополнительные плюшки для списков
%  например \begin{enumerate}[resume] позволяет продолжить нумерацию в новом списке
\usepackage{caption}

\usepackage{physics}

\usepackage{hyperref} % гиперссылки

\usepackage{multicol} % текст в несколько столбцов


\usepackage{fancyhdr} % весёлые колонтитулы
\pagestyle{fancy}
\lhead{Time Series and Stochastic Processes}
\chead{Final exam}
\rhead{2021-04-13, Rock 'N' Roll day}
\lfoot{}
\cfoot{DON'T PANIC}
\rfoot{}
\renewcommand{\headrulewidth}{0.4pt}
\renewcommand{\footrulewidth}{0.4pt}

\let\P\relax
\DeclareMathOperator{\P}{\mathbb{P}}
\DeclareMathOperator{\Cov}{\mathbb{C}ov}
\DeclareMathOperator{\E}{\mathbb{E}}
\DeclareMathOperator{\Var}{\mathbb{V}ar}
\newcommand{\cN}{\mathcal{N}}

\usepackage{todonotes} % для вставки в документ заметок о том, что осталось сделать
% \todo{Здесь надо коэффициенты исправить}
% \missingfigure{Здесь будет Последний день Помпеи}
% \listoftodos - печатает все поставленные \todo'шки


% более красивые таблицы
\usepackage{booktabs}
% заповеди из докупентации:
% 1. Не используйте вертикальные линни
% 2. Не используйте двойные линии
% 3. Единицы измерения - в шапку таблицы
% 4. Не сокращайте .1 вместо 0.1
% 5. Повторяющееся значение повторяйте, а не говорите "то же"



\usepackage{fontspec}
\usepackage{polyglossia}

\setmainlanguage{english}
\setotherlanguages{russian}

% download "Linux Libertine" fonts:
% http://www.linuxlibertine.org/index.php?id=91&L=1
\setmainfont{Linux Libertine O} % or Helvetica, Arial, Cambria
% why do we need \newfontfamily:
% http://tex.stackexchange.com/questions/91507/
\newfontfamily{\cyrillicfonttt}{Linux Libertine O}

\AddEnumerateCounter{\asbuk}{\russian@alph}{щ} % для списков с русскими буквами
% \setlist[enumerate, 2]{label=\asbuk*),ref=\asbuk*}




\begin{document}


\textbf{Estimation questions}

\begin{enumerate}


    \item To go to the mountain top I use a gondola lift in the morning. 
    I go back from the top using the same gondola lift in the evening. 
    Cabins are numbered from $1$ to $a$. 

    I have noticed that the absolute difference of cabin numbers of my two trips was $10$. 

    \begin{enumerate}
        \item Estimate $a$ using maximum likelihood. 
        \item Estimate $a$ using method of moments. 
    \end{enumerate}

    \item Random variables $X_1$, $X_2$, \ldots,  $X_n$ are independent identically distributed with density 
    \[
    f(x_i \mid \lambda, a) = \frac{\lambda}{2} \exp(-\lambda \abs{x_i - a}).    
    \]

    Observed values for $n=3$ are $-3$, $1$, $11$.

    \begin{enumerate}
        \item Estimate $\lambda$ using method of moments for fixed $a = 1$. 
        \item Estimate $\lambda$ and $a$ using maximum likelihood.
    \end{enumerate}

    \item Random variables $X_1$, \ldots, $X_n$ are independent and normally distributed $\cN(1, 1/b)$. 
    
    \begin{enumerate}
        \item Estimate $b$ using maximum likelihood.
        \item Does the estimator achive the Cramer-Rao lower bound?
        \item Is the estimator consistent?
        \item Is the estimator unbiased?
    \end{enumerate}

    \item Random variables $X_1$, $X_2$, \ldots,  $X_n$ are independent identically distributed with density 
    \[
    f(x_i \mid \lambda) = \frac{\lambda}{2} \exp(-\lambda \abs{x_i}).    
    \]

    For $n=100$ I have 40 negative values with sum equal to $-300$ and 60 positive values with sum equal to $500$. 

    \begin{enumerate}
        \item Test the hypothesis $\lambda = 1$ using LR approach at significance level $\alpha=0.01$.
        \item Test the hypothesis $\lambda = 1$ using LM approach at significance level $\alpha=0.01$.
    \end{enumerate}


\end{enumerate}

\newpage
    \textbf{Distribution questions}

    \begin{enumerate}[resume]
    \item I have three problems in the home assignment. 
    Time spent on each problem is modelled by independend exponentially distributed random variables with rate $\lambda$: $X_1$, $X_2$, $X_3$.

    \begin{enumerate}
        \item Find the moment generating function of $X_i$ and hence the moment generating function of $S = X_1 + X_2 + X_3$.
        \item Find $\E(S^3)$.
        \item Find the joint density of $R = X_1 / (X_1 + X_2 + X_3)$ and $S$.
    \end{enumerate}

    \item I have $100$ numbers written on small sheets of paper: $x_1$, $x_2$, \ldots, $x_{100}$. The sum of these numbers is $1$. 
    
    Find the possible values of the sum 
    \[
    \frac{x_1}{\sqrt{1-x_1}} +     \frac{x_2}{\sqrt{1-x_2}} + \ldots + \frac{x_{100}}{\sqrt{1-x_{100}}}.
    \]
    

    Hint: consider a randomly selected number $X$ and apply the Jensen's inequality.
    
 
\end{enumerate}


\end{document}
