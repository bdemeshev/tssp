\documentclass[12pt]{article}

\usepackage{tikz} % картинки в tikz
\usepackage{microtype} % свешивание пунктуации

\usepackage{array} % для столбцов фиксированной ширины

\usepackage{indentfirst} % отступ в первом параграфе

\usepackage{sectsty} % для центрирования названий частей
\allsectionsfont{\centering}

\usepackage{amsmath, amsfonts, amssymb} % куча стандартных математических плюшек

\usepackage{comment}

\usepackage[top=2cm, left=1.2cm, right=1.2cm, bottom=2cm]{geometry} % размер текста на странице

\usepackage{lastpage} % чтобы узнать номер последней страницы

\usepackage{enumitem} % дополнительные плюшки для списков
%  например \begin{enumerate}[resume] позволяет продолжить нумерацию в новом списке
\usepackage{caption}


\usepackage{fancyhdr} % весёлые колонтитулы
\pagestyle{fancy}
\lhead{Time Series and Stochastic Processes}
\chead{}
\rhead{HA}
\lfoot{2020-2021}
\cfoot{}
\rfoot{\thepage/\pageref{LastPage}}
\renewcommand{\headrulewidth}{0.4pt}
\renewcommand{\footrulewidth}{0.4pt}



\usepackage{todonotes} % для вставки в документ заметок о том, что осталось сделать
% \todo{Здесь надо коэффициенты исправить}
% \missingfigure{Здесь будет Последний день Помпеи}
% \listoftodos - печатает все поставленные \todo'шки


% более красивые таблицы
\usepackage{booktabs}
% заповеди из докупентации:
% 1. Не используйте вертикальные линни
% 2. Не используйте двойные линии
% 3. Единицы измерения - в шапку таблицы
% 4. Не сокращайте .1 вместо 0.1
% 5. Повторяющееся значение повторяйте, а не говорите "то же"



\usepackage{fontspec}
\usepackage{polyglossia}

\setmainlanguage{english}
\setotherlanguages{english}

% download "Linux Libertine" fonts:
% http://www.linuxlibertine.org/index.php?id=91&L=1
\setmainfont{Linux Libertine O} % or Helvetica, Arial, Cambria
% why do we need \newfontfamily:
% http://tex.stackexchange.com/questions/91507/
\newfontfamily{\cyrillicfonttt}{Linux Libertine O}

%\AddEnumerateCounter{\asbuk}{\russian@alph}{щ} % для списков с русскими буквами
%\setlist[enumerate, 2]{label=\asbuk*),ref=\asbuk*}

%% эконометрические сокращения
\DeclareMathOperator{\Cov}{\mathbb{C}ov}
\DeclareMathOperator{\Corr}{\mathbb{C}orr}
\DeclareMathOperator{\Var}{\mathbb{V}ar}
\DeclareMathOperator{\E}{\mathbb{E}}
\DeclareMathOperator{\tr}{trace}
\DeclareMathOperator{\card}{card}

\newcommand \hb{\hat{\beta}}
\newcommand \hs{\hat{\sigma}}
\newcommand \htheta{\hat{\theta}}
\newcommand \s{\sigma}
\newcommand \hy{\hat{y}}
\newcommand \hY{\hat{Y}}
\newcommand \e{\varepsilon}
\newcommand \he{\hat{\e}}
\newcommand \z{z}
\newcommand \hVar{\widehat{\Var}}
\newcommand \hCorr{\widehat{\Corr}}
\newcommand \hCov{\widehat{\Cov}}
\newcommand \cN{\mathcal{N}}
\newcommand \RR{\mathbb{R}}
\newcommand \NN{\mathbb{N}}
\newcommand{\cF}{\mathcal{F}}
\newcommand{\cH}{\mathcal{H}}


\begin{document}

\section*{Home Assignment 1}

\begin{enumerate}
  \item Let $\Omega = \RR$. Explicitely find the sigma-algebras $\cF_1 = \sigma(A)$, $\cF_2 = \sigma(B)$, $\cF_3 = \sigma(A, B)$ where $A=[-10;5]$ and $B=(0;10)$.
  \item I throw a die once. Let $X$ be the result of the toss. 
  Count the number of events in sigma-algebras $\cF_1 =\sigma (X)$, $\cF_2 = \sigma (\{X>3\})$, $\cF_3 = \sigma (\{X > 3\}, \{X<5\})$.

  \item Let $\Omega = \RR$. The sigma-algebra $\cF$ is generated by all the sets of the form $(-\infty, t]$,
  \[
  \cF = \sigma \left( \{ (-\infty; t] \mid t \in \RR\} \right)  
  \]
  Check whether $A_1 = (0; 10) \in \cF$, $A_2 = \{5\} \in \cF$, $A_3 = \NN \in \cF$. 

  \item Prove the following statements or provide a counter-example:
  \begin{enumerate}
    \item If $\cF_1$ and $\cF_2$ are sigma-algebras then $\cF = \cF_1 \cup \cF_2$ is sigma-algebra.
    \item If $X$ and $Y$ are independent random variables then $\card \sigma(X, Y) = \card \sigma(X) + \card \sigma(Y)$.
  \end{enumerate}

  \item I throw a die infinite number of times. Let $X_n$ be the result of the $n$-th toss. 
  Consider a pack of sigma-algebras: $\cF_n = \sigma (X_1, \ldots, X_n)$ and $\cH_n = \sigma (X_n, X_{n+1}, X_{n+2}\, \ldots)$.

  Provide and example of an event such that 
  \begin{enumerate}
    \item $A_1 \in \cF_{2020}$;
    \item $A_2 \in \cH_{2020}$;
    \item $A_3 \in \cF_{2020}$ and $A_3 \in \cH_{2020}$;
    \item $A_4 \in \cF_n$ for all $n$, $A_4 \in \cH_n$ for all $n$, $A_4 \neq \emptyset$, $A_4 \neq \Omega$.
  \end{enumerate}

\end{enumerate}


Deadline: 25 September 2020, 21:00 MSK.


\newpage
\section*{Home Assignment 2}
\begin{enumerate}
  \item 
\end{enumerate}




\end{document}
